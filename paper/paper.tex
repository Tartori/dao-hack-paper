\documentclass[a4paper, 11pt]{scrartcl}

\usepackage{csquotes}
%%\usepackage[french]{babel}
%%\usepackage[ngerman]{babel}


%% Choose default font for the document
%% Warning : only ONE of the following should be enabled
\usepackage{kpfonts}
%%\usepackage{libertine}

%% The following chose the default language for the document and
%% use the default typography rules for the choosen language.
\usepackage{polyglossia}
\setdefaultlanguage{english}
%% \setdefaultlanguage{german}
%%\setdefaultlanguage{french}

\usepackage[backend=biber, style=ieee]{biblatex}
\addbibresource{paper.bib}

\usepackage{graphicx}


%% to cite use \cite{Jerald:2015:VBH:2792790}
\begin{document}
\title{The DAO Hack}
\date{\today}   %% or \date{01 november 2018}
\author{
  Ramon Boss (\texttt{ramon.boss@students.bfh.ch}) \\
  Julian Stampfli (\texttt{julianjimmy.stampfli@students.bfh.ch}) 
}
 
\maketitle

\paragraph{Abstract}

In this paper we will explain what the DAO is, what happened to it and how the DAO could be exploited. We will give an overview of the security bug and on the fallout that followed.

\setcounter{tocdepth}{2}
\tableofcontents
\clearpage

\section{Introduction}

To properly understand what has happened, first one needs to have a proper foundation on which to build upon. This first section will be composed of those foundations. If you can't be bothered by this you can skip directly to the next section. 

\subsection{Blockchain}

The blockchain is a sequence of interlocked blocks of data. Those block essentialy contain the link to a parent block and a ennumeration of transactions. They are generated by the ethereum network and as such, nobody is in control of the blockchain itself. Decisions have to be made by the majority of the network or else the network would split into two different sub-chains. \cite{blockchainKai}

A benefit of the blockchain is that no single party can change past transactions. Thus, once something is in the blockchain, it is very hard to get rid of it. \cite{blockchainKai}

\subsubsection{Split / Hard Fork}

The Blockchain can be split at all times. Meaning everybody can take the chain and create a new block for any older parent block. Mostly smaller splits are simply ignored because they are non-consequential. However, when a larger part of the network decides that they want to change something, they could take the blockchain as it was at any point in time and create a new chain which has it's own history and it's own cryptocurrency. \cite{blockchainKai}

\subsubsection{Smart Contract / Transaction}

In the traditional sense a money transaction states who pays whom how much. The transactions in the blockchain can do just that, but mostly they have more functionality. For instance, to verify that the sender actually is who he says needs to append a signature to the transaction. Without this anybody could post transactions in the name of others. This signature has to be verified by all the nodes because invalid transactions should never end up in the blockchain. \cite{blockchainKai}

This is the simplest form of a smart contract. In comparison to a traditional transaction it contains of logic in addition to the data. This logic is implemented with a simple scripting language for most blockchains. \cite{blockchainKai} 

The validation of a signature is not the only usecase for smart contracts. A transaction could be bound to an arbitary amount of rules that can be represented by the scripting language. \cite{blockchainKai}

\subsection{Ethereum}

Ethereum is one type of a cryptocurrency. Its main difference to others like Bitcoin is that it can run a complex programming language called Solidity. With Solidity you can create loops, functions and recursion which then are run on the ethereum blockchain. This has huge advantages as one can create powerful smart contracts and bind the payment to those. Only with such a powerful language the DAO could be created. \cite{eth}

\subsection{The DAO}



\section{What happened?}

In May 2016 when the DAO was launched on the ethereum blockchain it was funded with 12.7M ether. As one ether was worth about 10-12 Dollars at that time this corresponds to about 150 million dollars. 

Soon after, on June 18 2016, an unknown party was able to drain the funds of the DAO by exploiting a security bug that was present in the function that is used to withdraw the ether. He was able to retrieve 3.5M ether which at that times would have been around 45 million dollars. 

\subsection{How was that possible?}



\section{What was the fallout?}

\section{Conclusion}

\nocite{*}
\clearpage
%% Print the bibibliography and add the section to the table of content
\printbibliography[heading=bibintoc]

\end{document}
