%!TEX program = lualatex
\documentclass[a4paper, 11pt]{scrartcl}

\usepackage{csquotes}
%%\usepackage[french]{babel}
%%\usepackage[ngerman]{babel}


%% Choose default font for the document
%% Warning : only ONE of the following should be enabled
\usepackage{kpfonts}
%%\usepackage{libertine}

%% The following chose the default language for the document and
%% use the default typography rules for the choosen language.
\usepackage{polyglossia}
\setdefaultlanguage{english}
%% \setdefaultlanguage{german}
%%\setdefaultlanguage{french}

\usepackage[backend=biber, style=ieee]{biblatex}
\addbibresource{paper.bib}

\usepackage{graphicx}


%% to cite use \cite{Jerald:2015:VBH:2792790}
\begin{document}
\title{The DAO Hack}
\date{\today}   %% or \date{01 november 2018}
\author{
  Ramon Boss (\texttt{ramon.boss@students.bfh.ch}) \\
  Julian Stampfli (\texttt{julianjimmy.stampfli@students.bfh.ch}) 
}
 
\maketitle

\paragraph{Abstract}
The DAO (decentralized autonomous organization) was an investment idea built on the ethereum blockchain. It was launched at the beginning of May 2016, had a lot of investors and raised a lot of money in ether. 

Due to a security bug an unknown investor was able to retrieve a large portion of this investment in mid June 2016. 

In this paper we will shortly explain what the DAO was meant to be and how it could be exploited. We will focus mostly on the security bug and the fallout.

\setcounter{tocdepth}{2}
\tableofcontents
\clearpage

\section{DAO}

\section{Conclusion}

\nocite{*}
\clearpage
%% Print the bibibliography and add the section to the table of content
\printbibliography[heading=bibintoc]

\end{document}
