%!TEX program = lualatex
\documentclass[a4paper, 11pt]{scrartcl}

\usepackage{csquotes}
%%\usepackage[french]{babel}
%%\usepackage[ngerman]{babel}


%% Choose default font for the document
%% Warning : only ONE of the following should be enabled
\usepackage{kpfonts}
%%\usepackage{libertine}

%% The following chose the default language for the document and
%% use the default typography rules for the choosen language.
\usepackage{polyglossia}
\setdefaultlanguage{english}
%% \setdefaultlanguage{german}
%%\setdefaultlanguage{french}

\usepackage[backend=biber, style=ieee]{biblatex}
\addbibresource{paper.bib}

\usepackage{graphicx}


%% to cite use \cite{Jerald:2015:VBH:2792790}
\begin{document}
\title{The DAO Hack}
\date{\today}   %% or \date{01 november 2018}
\author{
  Ramon Boss (\texttt{ramon.boss@students.bfh.ch}) \\
  Julian Stampfli (\texttt{julianjimmy.stampfli@students.bfh.ch})
}

\maketitle

\paragraph{Abstract}

In this paper we will explain what the DAO is, what happened to it and how the DAO could be exploited.
We will give an overview of the security bug and on the fallout that followed.

\setcounter{tocdepth}{2}
\tableofcontents
\clearpage

\section{Introduction}

To properly understand what has happened, first one needs to have a proper foundation on which to build upon.
This first section will be composed of those foundations.
If you can't be bothered by this you can skip directly to the next section.

\subsection{Blockchain}

The blockchain is a sequence of interlocked blocks of data.
Those block essentially contain the link to a parent block and an enumeration of transactions.
They are generated by the ethereum network and as such, nobody is in control of the blockchain itself.
Decisions have to be made by the majority of the network or else the network would split into two different sub-chains. \cite{blockchainKai}

A benefit of the blockchain is that no single party can change past transactions.
Thus, once something is in the blockchain, it is very hard to get rid of it. \cite{blockchainKai}

\subsubsection{Split / Hard Fork}

The Blockchain can be split at all times.
Meaning everybody can take the chain and create a new block for any older parent block.
Mostly smaller splits are simply ignored because they are non-consequential.
However, when a larger part of the network decides that they want to change something, they could take the blockchain as it was at any point in time and create a new chain which has its own history and its own cryptocurrency. \cite{blockchainKai}

\subsubsection{Smart Contract / Transaction}

In the traditional sense a money transaction states who pays whom how much.
The transactions in the blockchain can do just that, but mostly they have more functionality.
For instance, to verify that the sender actually is who he says needs to append a signature to the transaction.
Without this anybody could post transactions in the name of others.
This signature has to be verified by all the nodes because invalid transactions should never end up in the blockchain. \cite{blockchainKai}

This is the simplest form of a smart contract.
In comparison to a traditional transaction it contains of logic in addition to the data.
This logic is implemented with a simple scripting language for most blockchains. \cite{blockchainKai}

The validation of a signature is not the only use case for smart contracts.
A transaction could be bound to an arbitrary amount of rules that can be represented by the scripting language. \cite{blockchainKai}

\subsection{Ethereum}

Ethereum is one type of cryptocurrency.
The cash is calles ether.
Its main difference to others like Bitcoin is that it can run a complex programming language called Solidity.
With Solidity, you can create loops, functions and recursion which then are run on the ethereum blockchain.
This has huge advantages as one can create powerful smart contracts and bind the payment to those.
Only with such a powerful language the DAO could be created. \cite{eth}

\subsection{Investment Fund}

An investment fund is a fund where investors can invest money.
With the money the fund manager manages which shares to buy/sell when based on the defined goals, risks and other factors of the specific fund.
The gain can then be payed out to the investors and the loss will shrink the fund.

\subsection{The DAO}

The DAO is an open source software written in Solidity build on the ethereum blockchain.

The DAO is like an investment fund running without any hierarchical management.
Investors (ethereum addresses) can exchange ether for DAO tokens (the DAOs currency) in the initial phase.

The initial phase defines some starting properties of the DAO like the minimum ether required to successfully build the DAO.
Instead of a fund manager like in the investment fund each investor gets voting rights based on the invested money.
With this voting rights one can vote where the money should be invested.
To propose an investment an investor can create a proposal.
The proposal contains the recipient (an ethereum address) of the investment, the amount in ether, the voting deadline, among other properties.
When a proposal is accepted, which means more than a given amount of votes voted yes the DAO executes the proposal, meaning it transfers the money to the recipient.
To prevent proposal spams the proposal creator has to pay a small fee for each proposal he creates, which he gets back iff the proposal is accepted.

But how do we prevent illegal proposals or proposals that pays out all the investment to a bad person who has more then 50\% of the voting rights?
For this reasons each DAO has the curator (an ethereum address), a person who can accept or reject proposals before the voting window gets opened.
This curator is also defined in the initial phase.

The curator principle removes a bit of the non hierarchical management.
To give the investors more flexibility they can change the curator at any time for any reason.
For this purpose one has to create a special proposal called a split proposal.
The creator defines a new curator in this proposal and the old curator can not reject it.
The DAO automatically starts the voting period.

After this period the DAO is split in two.
All investors voting yes are now in the new DAO and their investment is transfered to it while the others stay in the old one.

\section{What happened?}

In May 2016 when the DAO was launched on the ethereum blockchain it was funded with 12.7M ether.
As one ether was worth about 10-12 Dollars at that time, this corresponds to about 150 million dollars.

Soon after, on June 18, 2016, an unknown party was able to drain the funds of the DAO by exploiting a security bug that was present in the function that is used to withdraw the ether.
He was able to retrieve 3.5M ether which at that times would have been around 45 million dollars.

\subsection{How was that possible?}

The DAO includes the already mentioned functionality to split the DAO.
This was intended for people who either want to retrieve their funds or who weren't happy with the decisions of the DAO and wanted to invest differently.
Everybody could split the DAO at any time with a one-week waiting period.

The split function contains a function called withdrawRewardFor which calls rewardAccount.payOut which in turn calls \textunderscore recipient.call.value. The first two functions are defined in the DAO code and as such can't be changed at will. The call to recipient however is in the attackers hands. He can do there whatever he desires. In code this looks like that: \cite{deconstructingDaoAttack}

\begin{verbatim}
  function withdrawRewardFor(address _account) noEther 
    internal returns (bool _success) {
        if ((balanceOf(_account) * rewardAccount.accumulatedInput()) /
          totalSupply < paidOut[_account])
            throw;

        uint reward =
            (balanceOf(_account) * rewardAccount.accumulatedInput()) /
              totalSupply - paidOut[_account];
        if (!rewardAccount.payOut(_account, reward))
            throw;
        paidOut[_account] += reward;
        return true;
    }

    function payOut(address _recipient, uint _amount) returns (bool) {
        if (msg.sender != owner || msg.value > 0 ||
          (payOwnerOnly && _recipient != owner))
            throw;
        if (_recipient.call.value(_amount)()) {
            PayOut(_recipient, _amount);
            return true;
        } else {
            return false;
        }
    }
\end{verbatim}

One remark to this, some days before the attack there was a security fix proposed to the DAO master on Github, but wasn't committed immediately. With this fix the attack would not have been possible in this form. \cite{securityFixPayout}

Event though this had flaws, it alone wouldn't have been enough for this particular exploit. One more glaring flaw is in the splitDAO function itself: \cite{deconstructingDaoAttack}

\begin{verbatim}
        // Burn DAO Tokens
        Transfer(msg.sender, 0, balances[msg.sender]);
        withdrawRewardFor(msg.sender); // be nice, and get his rewards
        totalSupply -= balances[msg.sender];
        balances[msg.sender] = 0;
        paidOut[msg.sender] = 0;
        return true;
\end{verbatim}

As you can see from this snippet which comes from the splitDAO function, the withdrawRewardFor function gets called and then the balance of the sender gets set to 0. This way, if someone can call splitDAO recursively, withdrawRewardFor will get called multiple times with the same arguments. With this anybody could basically retrieve the whole investment by recursively calling splitDAO. However, it is important to note that if the call stack gets to big, it might fail due to a stack overflow or other issues. Thus, one attacker can only retrieve a limited amount of ether. The call stack would look something like this: \cite{deconstructingDaoAttack}

\begin{verbatim}
splitDAO:
  withdrawRewardFor:
    rewardAccount.payOut:
      _recipient.call.value:
        splitDAO:
          withdrawRewardFor:
            rewardAccount.payOut:
              _recipient.call.value:
                ...  
                  splitDAO:
                    withdrawRewardFor:
                      rewardAccount.payOut:
                        _recipient.call.value
\end{verbatim}

After the attacker exploited this he had a child DAO with more funds than it should have and a parent DAO which doesn't suspect that anything went wrong. To make things even worse, he was able to send back tokens to restart the attack. Only this way he was able to get this many ether out of it. \cite{deconstructingDaoAttack}

\section{What was the fallout?}



\section{Conclusion}

\nocite{*}
\clearpage
%% Print the bibibliography and add the section to the table of content
\printbibliography[heading=bibintoc]

\end{document}
